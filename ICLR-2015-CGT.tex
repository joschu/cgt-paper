\documentclass{article} % For LaTeX2e
\usepackage{iclr2016_conference,times}
\usepackage{hyperref}
\usepackage{url}


\title{Computation Graph Toolkit}


\author{
Authors TBD
% Antiquus S.~Hippocampus, Natalia Cerebro \& Amelie P. Amygdale \thanks{ Use footnote for providing further information
% about author (webpage, alternative address)---\emph{not} for acknowledging
% funding agencies.  Funding acknowledgements go at the end of the paper.} \\
% Department of Computer Science\\
% Cranberry-Lemon University\\
% Pittsburgh, PA 15213, USA \\
% \texttt{\{hippo,brain,jen\}@cs.cranberry-lemon.edu} \\
% \And
% Ji Q. Ren \& Yevgeny LeNet \\
% Department of Computational Neuroscience \\
% University of the Witwatersrand \\
% Joburg, South Africa \\
% \texttt{\{robot,net\}@wits.ac.za} \\
% \AND
% Coauthor \\
% Affiliation \\
% Address \\
% \texttt{email}
}

% The \author macro works with any number of authors. There are two commands
% used to separate the names and addresses of multiple authors: \And and \AND.
%
% Using \And between authors leaves it to \LaTeX{} to determine where to break
% the lines. Using \AND forces a linebreak at that point. So, if \LaTeX{}
% puts 3 of 4 authors names on the first line, and the last on the second
% line, try using \AND instead of \And before the third author name.

\newcommand{\fix}{\marginpar{FIX}}
\newcommand{\new}{\marginpar{NEW}}

%\iclrfinalcopy % Uncomment for camera-ready version

\begin{document}


\maketitle

\begin{abstract}

Efficient evaluation of mathematical functions and their derivatives is a primary need in scientific computing.
Building on ideas from the recent software library called Theano, we develop a new library called the Computation Graph Toolkit (CGT).
CGT provides automatic differentiation, and uses a fast compilation pipeline that enables parallel computation, 
with concurrent usage of multiple CPUs and GPUs.
This paper describes CGTs methodology for automatic differentiation, computation graph simplification, code generation, and numerical evaluation.
We provide benchmarks that compare the performance of CGT with previous libraries such as Torch and Theano.

\end{abstract}

\section{Introduction}

first part:
(1) describe how symbolic autodiff is used in machine learning. 
(2) describe existing libraries and their functionality.
(3) describe their limitations

second part:
(4) how does cgt overcome their limitations
(5) brief summary of architecture with emphasis on new parts
(6) preview of comparisons

\section{Symbolic Automatic Differentiation}

(1) define computation graph
(2) explain concept of pullback and how to differentiate a computation graph
(3) also describe pushforward
(4) compare cgt's SSA datastructure to Theano's factor graph / bipartite graph
(5) compare CGT/Theano approach to Torch/Caffe approach with pros/cons

\section{asdf}

\subsubsection*{Acknowledgments}

Use unnumbered third level headings for the acknowledgments. All
acknowledgments, including those to funding agencies, go at the end of the paper.

\bibliography{iclr2016_conference}
\bibliographystyle{iclr2016_conference}

\end{document}
